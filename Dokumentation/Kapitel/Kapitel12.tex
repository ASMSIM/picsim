\section{Vor- und Nachteile eines Simulators}

Vorteile:
Durch eine Simulation können Versuche die unter gefährlichen Umständen stattfinden müssen sicher nachgestellt werden (z.B. Crash-Simulationen mit Autos und Crash-Test-Dummys).
Aber auch Versuche die aus Kostengründen in der Realität oftmals schwierig nachzustellen sind können durch Simulationen begrenzt ersetzt werden.
Durch den verlangsamten Ablauf einer Simulation sind außerdem Fehler oder Ergebnisse leichter nachzuvollziehen als in der Wirklichkeit.
Im Falle des Mikrocontrollers können Programme vor ihrem praktischen Einsatz getestet und debuggt werden um so mögliche Fehler im Praxiseinsatz frühzeitig zu erkennen und auszubessern.

Nachteile:
Eine Simulation ist meist durch begrenzte Ressourcen eingeschränkt. Sei es die Rechenleistung einer Computersimulation oder Geld und Zeit die für eine Simulation eingesetzt werden müssen. Oftmals wird deswegen nur ein vereinfachtes Modell der Wirklichkeit eingesetzt. Durch diese Vereinfachung kann es zu ungenauen Messergebnisse oder Situationen kommen die in der Realität vielleicht gar nicht vorkommen.
Für den PIC16-Simulatior ist es wichtig möglichst fehlerfrei und genau zu arbeiten da Fehler innerhalb der Simulation auf falsche Rückschlüsse auf das für den Mikrocontroller entwickelte Programm führen könnte. Auch zu bedenken ist es das die Laufzeit in der Simulation nicht der Realzeit entspricht und somit das Programm in der Realität schneller sein würde.