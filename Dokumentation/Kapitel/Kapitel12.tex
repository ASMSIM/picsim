\section{Vor- und Nachteile eines Simulators}

\textbf{Vorteile:}
Durch eine Simulation k\"onnen Versuche die unter gef\"ahrlichen Umst\"anden stattfinden m\"ussen sicher nachgestellt werden (z.B. Crash-Simulationen mit Autos und Crash-Test-Dummys).
Aber auch Versuche die aus Kostengr\"unden in der Realit\"at oftmals schwierig nachzustellen sind k\"onnen durch Simulationen begrenzt ersetzt werden.
Durch den verlangsamten Ablauf einer Simulation sind au\ss erdem Fehler oder Ergebnisse leichter nachzuvollziehen als in der Wirklichkeit.
Im Falle des Mikrocontrollers k\"onnen Programme vor ihrem praktischen Einsatz getestet und debuggt werden um so m\"ogliche Fehler im Praxiseinsatz fr\"uhzeitig zu erkennen und auszubessern.\\
\\
\textbf{Nachteile:}
Eine Simulation ist meist durch begrenzte Ressourcen eingeschr\"ankt. Sei es die Rechenleistung einer Computersimulation oder Geld und Zeit die f\"ur eine Simulation eingesetzt werden m\"ussen. Oftmals wird deswegen nur ein vereinfachtes Modell der Wirklichkeit eingesetzt. Durch diese Vereinfachung kann es zu ungenauen Messergebnisse oder Situationen kommen die in der Realit\"at vielleicht gar nicht vorkommen.
F\"ur den PIC16-Simulatior ist es wichtig m\"oglichst fehlerfrei und genau zu arbeiten da Fehler innerhalb der Simulation auf falsche R\"uckschl\"usse auf das f\"ur den Mikrocontroller entwickelte Programm f\"uhren k\"onnte. Auch zu bedenken ist es das die Laufzeit in der Simulation nicht der Realzeit entspricht und somit das Programm in der Realit\"at schneller sein w\"urde.