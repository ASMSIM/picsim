\section{Mikrocontroller PIC16F84A}

Ein Mikrocontroller ist eine Art Mikrorechnersystem, bei welchem neben ROM und RAM auch Peripherieeinheiten wie Schnittstellen, Timer und Bussysteme auf einem einzigen Chip integriert sind.
Die Hauptanwendungsgebiete sind die Steuerungs-, Mess- und Regelungstechnik, sowie die Kommunikationstechnik und die Bildverarbeitung. Mikrocontroller sind in der Regel in Embedded Systems, in die Anwendung eingebettete Systeme, und somit in der Regel von au\ss en nicht sichtbar. Ebenso verf\"ugen sie, im Gegensatz zum PC, nicht \"uber eine direkte Bedien- und Prorgrammierschnittstelle zum Benutzer. Sie werden in der Regel einmal programmiert und installiert.

Der PIC16F84 Mikrocontroller ist ein 8 Bit Mikrocontroller mit RISC-Architektur (Reduced-Instruction-Set-Computing). Es wird also auf komplexe Befehle verzichtet und mit jedem Befehl kann auf jedes Register zugegriffen werden. Der Mikrocontroller besitzt durch die eingesetzte Harvard-Architektur bis zu 14 Bit gro\ss e Befehle w\"ahrend die Gr\"o\ss e des separaten Datenbusses nur 8 Bit betr\"agt.

\begin{figure}[htb]
\centering
\includegraphics{Bilder/Harvard}
\caption{Harvard-Architektur}
\end{figure}

\newpage
Durch die Architektur ben\"otigen fast alle Anweisungen nur einen Instruction Cycle (Abarbeitung eines Maschinenbefehls).
Der PIC16 besitzt einen Stack mit Speicherplatz f\"ur 8 Adressen sowie 2 externe und 2 interne Interrupt Quellen. Dar\"uber hinaus besitzt der Pic16F ein gro\ss es Register, welches in zwei B\"anke unterteilt ist. Das Umschalten der B\"anke erfolgt im Programmcode. Die Speicherbereiche k\"onnen auch direkt \"uber ihre Registeradresse angesprochen werden.
